% Options for packages loaded elsewhere
\PassOptionsToPackage{unicode}{hyperref}
\PassOptionsToPackage{hyphens}{url}
\documentclass[
]{article}
\usepackage{xcolor}
\usepackage[margin=1in]{geometry}
\usepackage{amsmath,amssymb}
\setcounter{secnumdepth}{5}
\usepackage{iftex}
\ifPDFTeX
  \usepackage[T1]{fontenc}
  \usepackage[utf8]{inputenc}
  \usepackage{textcomp} % provide euro and other symbols
\else % if luatex or xetex
  \usepackage{unicode-math} % this also loads fontspec
  \defaultfontfeatures{Scale=MatchLowercase}
  \defaultfontfeatures[\rmfamily]{Ligatures=TeX,Scale=1}
\fi
\usepackage{lmodern}
\ifPDFTeX\else
  % xetex/luatex font selection
\fi
% Use upquote if available, for straight quotes in verbatim environments
\IfFileExists{upquote.sty}{\usepackage{upquote}}{}
\IfFileExists{microtype.sty}{% use microtype if available
  \usepackage[]{microtype}
  \UseMicrotypeSet[protrusion]{basicmath} % disable protrusion for tt fonts
}{}
\makeatletter
\@ifundefined{KOMAClassName}{% if non-KOMA class
  \IfFileExists{parskip.sty}{%
    \usepackage{parskip}
  }{% else
    \setlength{\parindent}{0pt}
    \setlength{\parskip}{6pt plus 2pt minus 1pt}}
}{% if KOMA class
  \KOMAoptions{parskip=half}}
\makeatother
\usepackage{color}
\usepackage{fancyvrb}
\newcommand{\VerbBar}{|}
\newcommand{\VERB}{\Verb[commandchars=\\\{\}]}
\DefineVerbatimEnvironment{Highlighting}{Verbatim}{commandchars=\\\{\}}
% Add ',fontsize=\small' for more characters per line
\usepackage{framed}
\definecolor{shadecolor}{RGB}{248,248,248}
\newenvironment{Shaded}{\begin{snugshade}}{\end{snugshade}}
\newcommand{\AlertTok}[1]{\textcolor[rgb]{0.94,0.16,0.16}{#1}}
\newcommand{\AnnotationTok}[1]{\textcolor[rgb]{0.56,0.35,0.01}{\textbf{\textit{#1}}}}
\newcommand{\AttributeTok}[1]{\textcolor[rgb]{0.13,0.29,0.53}{#1}}
\newcommand{\BaseNTok}[1]{\textcolor[rgb]{0.00,0.00,0.81}{#1}}
\newcommand{\BuiltInTok}[1]{#1}
\newcommand{\CharTok}[1]{\textcolor[rgb]{0.31,0.60,0.02}{#1}}
\newcommand{\CommentTok}[1]{\textcolor[rgb]{0.56,0.35,0.01}{\textit{#1}}}
\newcommand{\CommentVarTok}[1]{\textcolor[rgb]{0.56,0.35,0.01}{\textbf{\textit{#1}}}}
\newcommand{\ConstantTok}[1]{\textcolor[rgb]{0.56,0.35,0.01}{#1}}
\newcommand{\ControlFlowTok}[1]{\textcolor[rgb]{0.13,0.29,0.53}{\textbf{#1}}}
\newcommand{\DataTypeTok}[1]{\textcolor[rgb]{0.13,0.29,0.53}{#1}}
\newcommand{\DecValTok}[1]{\textcolor[rgb]{0.00,0.00,0.81}{#1}}
\newcommand{\DocumentationTok}[1]{\textcolor[rgb]{0.56,0.35,0.01}{\textbf{\textit{#1}}}}
\newcommand{\ErrorTok}[1]{\textcolor[rgb]{0.64,0.00,0.00}{\textbf{#1}}}
\newcommand{\ExtensionTok}[1]{#1}
\newcommand{\FloatTok}[1]{\textcolor[rgb]{0.00,0.00,0.81}{#1}}
\newcommand{\FunctionTok}[1]{\textcolor[rgb]{0.13,0.29,0.53}{\textbf{#1}}}
\newcommand{\ImportTok}[1]{#1}
\newcommand{\InformationTok}[1]{\textcolor[rgb]{0.56,0.35,0.01}{\textbf{\textit{#1}}}}
\newcommand{\KeywordTok}[1]{\textcolor[rgb]{0.13,0.29,0.53}{\textbf{#1}}}
\newcommand{\NormalTok}[1]{#1}
\newcommand{\OperatorTok}[1]{\textcolor[rgb]{0.81,0.36,0.00}{\textbf{#1}}}
\newcommand{\OtherTok}[1]{\textcolor[rgb]{0.56,0.35,0.01}{#1}}
\newcommand{\PreprocessorTok}[1]{\textcolor[rgb]{0.56,0.35,0.01}{\textit{#1}}}
\newcommand{\RegionMarkerTok}[1]{#1}
\newcommand{\SpecialCharTok}[1]{\textcolor[rgb]{0.81,0.36,0.00}{\textbf{#1}}}
\newcommand{\SpecialStringTok}[1]{\textcolor[rgb]{0.31,0.60,0.02}{#1}}
\newcommand{\StringTok}[1]{\textcolor[rgb]{0.31,0.60,0.02}{#1}}
\newcommand{\VariableTok}[1]{\textcolor[rgb]{0.00,0.00,0.00}{#1}}
\newcommand{\VerbatimStringTok}[1]{\textcolor[rgb]{0.31,0.60,0.02}{#1}}
\newcommand{\WarningTok}[1]{\textcolor[rgb]{0.56,0.35,0.01}{\textbf{\textit{#1}}}}
\usepackage{longtable,booktabs,array}
\usepackage{calc} % for calculating minipage widths
% Correct order of tables after \paragraph or \subparagraph
\usepackage{etoolbox}
\makeatletter
\patchcmd\longtable{\par}{\if@noskipsec\mbox{}\fi\par}{}{}
\makeatother
% Allow footnotes in longtable head/foot
\IfFileExists{footnotehyper.sty}{\usepackage{footnotehyper}}{\usepackage{footnote}}
\makesavenoteenv{longtable}
\usepackage{graphicx}
\makeatletter
\newsavebox\pandoc@box
\newcommand*\pandocbounded[1]{% scales image to fit in text height/width
  \sbox\pandoc@box{#1}%
  \Gscale@div\@tempa{\textheight}{\dimexpr\ht\pandoc@box+\dp\pandoc@box\relax}%
  \Gscale@div\@tempb{\linewidth}{\wd\pandoc@box}%
  \ifdim\@tempb\p@<\@tempa\p@\let\@tempa\@tempb\fi% select the smaller of both
  \ifdim\@tempa\p@<\p@\scalebox{\@tempa}{\usebox\pandoc@box}%
  \else\usebox{\pandoc@box}%
  \fi%
}
% Set default figure placement to htbp
\def\fps@figure{htbp}
\makeatother
\setlength{\emergencystretch}{3em} % prevent overfull lines
\providecommand{\tightlist}{%
  \setlength{\itemsep}{0pt}\setlength{\parskip}{0pt}}
\usepackage{float}
\usepackage{bookmark}
\IfFileExists{xurl.sty}{\usepackage{xurl}}{} % add URL line breaks if available
\urlstyle{same}
\hypersetup{
  hidelinks,
  pdfcreator={LaTeX via pandoc}}

\author{}
\date{\vspace{-2.5em}}

\begin{document}

\begin{Shaded}
\begin{Highlighting}[]
\CommentTok{\# Lade notwendige Bibliothek}
\FunctionTok{library}\NormalTok{(readr)}
\FunctionTok{library}\NormalTok{(knitr)}

\CommentTok{\# Lade die CSV{-}Datei in ein DataFrame}
\NormalTok{file\_path }\OtherTok{\textless{}{-}} \StringTok{"../1\_Datenset/ursprüngliche/netflix\_titles.csv"}
\NormalTok{movies\_df }\OtherTok{\textless{}{-}} \FunctionTok{read\_csv}\NormalTok{(file\_path)}
\end{Highlighting}
\end{Shaded}

\begin{verbatim}
## Rows: 8807 Columns: 12
## -- Column specification --------------------------------------------------------
## Delimiter: ","
## chr (11): show_id, type, title, director, cast, country, date_added, rating,...
## dbl  (1): release_year
## 
## i Use `spec()` to retrieve the full column specification for this data.
## i Specify the column types or set `show_col_types = FALSE` to quiet this message.
\end{verbatim}

\begin{Shaded}
\begin{Highlighting}[]
\CommentTok{\# Erstelle ein neues DataFrame für die Tabelle}
\NormalTok{spalten }\OtherTok{\textless{}{-}} \FunctionTok{colnames}\NormalTok{(movies\_df)}
\NormalTok{beschreibung }\OtherTok{\textless{}{-}} \FunctionTok{c}\NormalTok{(}
    \StringTok{"Einzigartige ID der Show"}\NormalTok{, }
    \StringTok{"Film oder Serie"}\NormalTok{, }
    \StringTok{"Titel der Show"}\NormalTok{, }
    \StringTok{"Regisseur"}\NormalTok{, }
    \StringTok{"Besetzung"}\NormalTok{,}
    \StringTok{"Produktionsland"}\NormalTok{,}
    \StringTok{"Hinzufügedatum"}\NormalTok{,}
    \StringTok{"Veröffentlichungsjahr"}\NormalTok{,}
    \StringTok{"Altersfreigabe"}\NormalTok{,}
    \StringTok{"Laufzeit in Minuten oder Staffeln"}\NormalTok{,}
    \StringTok{"Genres"}\NormalTok{,}
    \StringTok{"Kurzbeschreibung"}
\NormalTok{)}
\NormalTok{relevanz }\OtherTok{\textless{}{-}} \FunctionTok{c}\NormalTok{(}
    \StringTok{"nein"}\NormalTok{,}
    \StringTok{"ja"}\NormalTok{,}
    \StringTok{"ja"}\NormalTok{,}
    \StringTok{"ja"}\NormalTok{,}
    \StringTok{"ja"}\NormalTok{,}
    \StringTok{"ja"}\NormalTok{,}
    \StringTok{"nein"}\NormalTok{,}
    \StringTok{"nein"}\NormalTok{,}
    \StringTok{"ja"}\NormalTok{,}
    \StringTok{"ja"}\NormalTok{,}
    \StringTok{"ja"}\NormalTok{,}
    \StringTok{"ja"}
\NormalTok{)}

\CommentTok{\# Kombiniere die Beschreibungen und Relevanzen in ein neues DataFrame}
\NormalTok{tabelle\_df }\OtherTok{\textless{}{-}} \FunctionTok{data.frame}\NormalTok{(}\AttributeTok{Spalten =}\NormalTok{spalten, }\AttributeTok{Beschreibung =}\NormalTok{ beschreibung, }\AttributeTok{Relevanz =}\NormalTok{ relevanz)}

\CommentTok{\# Zeige die Tabelle an}
\FunctionTok{kable}\NormalTok{(}
\NormalTok{  tabelle\_df,}
  \AttributeTok{align =} \FunctionTok{c}\NormalTok{(}\StringTok{"r"}\NormalTok{, }\StringTok{"l"}\NormalTok{, }\StringTok{"c"}\NormalTok{),}
  \AttributeTok{col.names =} \FunctionTok{c}\NormalTok{(}\StringTok{"Spalten"}\NormalTok{, }\StringTok{"Beschreibung"}\NormalTok{, }\StringTok{"Relevant?"}\NormalTok{),}
  \AttributeTok{caption =} \StringTok{"Tabelle der Beschreibungen und Relevanzen"}
\NormalTok{)}
\end{Highlighting}
\end{Shaded}

\begin{longtable}[]{@{}rlc@{}}
\caption{Tabelle der Beschreibungen und Relevanzen}\tabularnewline
\toprule\noalign{}
Spalten & Beschreibung & Relevant? \\
\midrule\noalign{}
\endfirsthead
\toprule\noalign{}
Spalten & Beschreibung & Relevant? \\
\midrule\noalign{}
\endhead
\bottomrule\noalign{}
\endlastfoot
show\_id & Einzigartige ID der Show & nein \\
type & Film oder Serie & ja \\
title & Titel der Show & ja \\
director & Regisseur & ja \\
cast & Besetzung & ja \\
country & Produktionsland & ja \\
date\_added & Hinzufügedatum & nein \\
release\_year & Veröffentlichungsjahr & nein \\
rating & Altersfreigabe & ja \\
duration & Laufzeit in Minuten oder Staffeln & ja \\
listed\_in & Genres & ja \\
description & Kurzbeschreibung & ja \\
\end{longtable}

\begin{Shaded}
\begin{Highlighting}[]
\CommentTok{\# Lade notwendige Bibliothek}
\FunctionTok{library}\NormalTok{(readr)}
\FunctionTok{library}\NormalTok{(knitr)}

\CommentTok{\# Lade die CSV{-}Datei in ein DataFrame}
\NormalTok{file\_path }\OtherTok{\textless{}{-}} \StringTok{"../1\_Datenset/erstellte/fertig/fertig.csv"}
\NormalTok{movies\_df }\OtherTok{\textless{}{-}} \FunctionTok{read\_csv}\NormalTok{(file\_path)}
\end{Highlighting}
\end{Shaded}

\begin{verbatim}
## Rows: 10176 Columns: 12
## -- Column specification --------------------------------------------------------
## Delimiter: ","
## chr (11): show_id, type, title, director, cast, country, agerating, duration...
## dbl  (1): release_year
## 
## i Use `spec()` to retrieve the full column specification for this data.
## i Specify the column types or set `show_col_types = FALSE` to quiet this message.
\end{verbatim}

\begin{Shaded}
\begin{Highlighting}[]
\CommentTok{\# Erstelle ein neues DataFrame für die Tabelle}
\NormalTok{spalten }\OtherTok{\textless{}{-}} \FunctionTok{colnames}\NormalTok{(movies\_df)}
\NormalTok{beschreibung }\OtherTok{\textless{}{-}} \FunctionTok{c}\NormalTok{(}
    \StringTok{"Einzigartige ID der Show"}\NormalTok{, }
    \StringTok{"Film oder Serie"}\NormalTok{, }
    \StringTok{"Titel der Show"}\NormalTok{, }
    \StringTok{"Regisseur"}\NormalTok{, }
    \StringTok{"Besetzung"}\NormalTok{,}
    \StringTok{"Produktionsland"}\NormalTok{,}
    \StringTok{"Veröffentlichungsjahr"}\NormalTok{,}
    \StringTok{"Altersfreigabe"}\NormalTok{,}
    \StringTok{"Laufzeit in Minuten oder Staffeln"}\NormalTok{,}
    \StringTok{"Genres"}\NormalTok{,}
    \StringTok{"Kurzbeschreibung"}\NormalTok{,}
    \StringTok{"Netflix oder Disney+"}
\NormalTok{)}
\NormalTok{relevanz }\OtherTok{\textless{}{-}} \FunctionTok{c}\NormalTok{(}
    \StringTok{"nein"}\NormalTok{,}
    \StringTok{"ja"}\NormalTok{,}
    \StringTok{"ja"}\NormalTok{,}
    \StringTok{"ja"}\NormalTok{,}
    \StringTok{"ja"}\NormalTok{,}
    \StringTok{"ja"}\NormalTok{,}
    \StringTok{"nein"}\NormalTok{,}
    \StringTok{"ja"}\NormalTok{,}
    \StringTok{"ja"}\NormalTok{,}
    \StringTok{"ja"}\NormalTok{,}
    \StringTok{"ja"}\NormalTok{,}
    \StringTok{"ja"}
\NormalTok{)}

\CommentTok{\# Kombiniere die Beschreibungen und Relevanzen in ein neues DataFrame}
\NormalTok{tabelle\_df }\OtherTok{\textless{}{-}} \FunctionTok{data.frame}\NormalTok{(}\AttributeTok{Spalten =}\NormalTok{spalten, }\AttributeTok{Beschreibung =}\NormalTok{ beschreibung, }\AttributeTok{Relevanz =}\NormalTok{ relevanz)}

\CommentTok{\# Zeige die Tabelle an}
\FunctionTok{kable}\NormalTok{(}
\NormalTok{  tabelle\_df,}
  \AttributeTok{align =} \FunctionTok{c}\NormalTok{(}\StringTok{"r"}\NormalTok{, }\StringTok{"l"}\NormalTok{, }\StringTok{"c"}\NormalTok{),}
  \AttributeTok{col.names =} \FunctionTok{c}\NormalTok{(}\StringTok{"Spalten"}\NormalTok{, }\StringTok{"Beschreibung"}\NormalTok{, }\StringTok{"Relevant?"}\NormalTok{),}
  \AttributeTok{caption =} \StringTok{"Tabelle der Beschreibungen und Relevanzen"}
\NormalTok{)}
\end{Highlighting}
\end{Shaded}

\begin{longtable}[]{@{}rlc@{}}
\caption{Tabelle der Beschreibungen und Relevanzen}\tabularnewline
\toprule\noalign{}
Spalten & Beschreibung & Relevant? \\
\midrule\noalign{}
\endfirsthead
\toprule\noalign{}
Spalten & Beschreibung & Relevant? \\
\midrule\noalign{}
\endhead
\bottomrule\noalign{}
\endlastfoot
show\_id & Einzigartige ID der Show & nein \\
type & Film oder Serie & ja \\
title & Titel der Show & ja \\
director & Regisseur & ja \\
cast & Besetzung & ja \\
country & Produktionsland & ja \\
release\_year & Veröffentlichungsjahr & nein \\
agerating & Altersfreigabe & ja \\
duration & Laufzeit in Minuten oder Staffeln & ja \\
listed\_in & Genres & ja \\
description & Kurzbeschreibung & ja \\
platform & Netflix oder Disney+ & ja \\
\end{longtable}

\end{document}
